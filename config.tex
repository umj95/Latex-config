% Custom Commands:
% Use \maketitle to print the title info on the first text page
% Use \maketitlepage to print a dedicated title page
% Use \plagiat to print a Plagiatserklärung on its own page
% After \tableofcontents a page break always follows
% Use \workscited to print the works cited page (new page)

%%%%%%%%%%%%%%%%%%%
%%%%%Variables%%%%%
%%%%%%%%%%%%%%%%%%%
    % Personal Info
    \author{***REMOVED*** ***REMOVED***} %Name
    \newcommand{\authormail}{} %Email
    \newcommand{\matrikelnr}{} %Matrikelnummer
    \newcommand{\authoraddress}{} %Addresse
    
    % Department & Correction Info
    \newcommand{\city}{} %Stadt (Für Plagiatserkl.)
    \newcommand{\university}{} %Uni
    \newcommand{\faculty}{} %Fakultät
    \newcommand{\department}{} %Institut
    \newcommand{\handindate}{} %Abgabedatum
    
    \newcommand{\tutor}{} %Betreuer*in
    %Use \maketitle for regular title and \maketitlepage for full title page
    \newcommand{\coursetype}{} %Kursart
    \newcommand{\coursetitle}{} %Name des Kurses
    \newcommand{\coursenumber}{} %Kursnummer
    
    % Paper Info
    \title{
        Religious Reform, Princely Patronage, and Josquin's \emph{Miserere mei, Deus}} %Titel der Arbeit
    \date{}
    \newcommand{\papertype}{} %Art der Arbeit
    \newcommand{\keywords}{} %Schlagwörter
    
    % input encoding & language
    %\newcommand{\mainlanguage}{ngerman} %Uncomment if main language is German
    \newcommand{\mainlanguage}{UKenglish} 
    \usepackage[ngerman, \mainlanguage]{babel}
%
% Literature & Bibliography Resources
%
\usepackage[babel]{csquotes}

\usepackage[style=mla,backend=biber]{biblatex}
    \addbibresource{sources.bib} %.bib-file für Literatur
%%%%%%%%%%%%%%%%%%%%%%%%
%%%%%%%%%%%%%%%%%%%%%%%%
%%%%%%%%%%%%%%%%%%%%%%%%
%
% Create Titlepage, Plagiatserklärung etc.
%
%%%%%%%%%%%Full Page Title%%%%%%%%%%%%%%%%%%%%
\makeatletter
    \newcommand{\maketitlepage}{
        \phantomsection
        \pdfbookmark[0]{Cover}{Cover}
        \label{Deckblatt}
        \thispagestyle{empty}
        \begin{titlepage}
        
        %Seite zentrieren
        %\begin{adjustwidth}{-2cm}{}
        \renewcommand{\thepage}{}
        
        \begin{center}
            \selectlanguage{ngerman}
            
            \ifdefined\university
                \large{\university\\}
            \else
                \\ \fi
             
            \ifdefined\faculty
                \large{\faculty\\}
            \else
                \\ \fi
            
            \ifdefined\department
                \large{\department\\}
            \else
                \\ \fi
            
            \vspace*{6em}
            
            \foreignlanguage{UKenglish}{
                \Huge{\textsc{\@title}}
            }
            
            \vspace*{3em}
            
            \ifdefined\papertype
                \Large{\papertype\\ \vspace{2em}}
            \else
                \\ \fi
            
            \ifdefined\tutor
                \huge{\tutor\\ \vspace*{1.5em}}
            \else
                \\ \fi
            
            \ifdefined\handindate
                \large{Abgabedatum: \handindate\\}
            \else
                \large{\today\\} \fi
            
            %\includegraphics[width=0.35\textwidth]{Unisiegel.png}}
            
        \end{center}
        \vfill
        
        \begin{flushleft}
        \@author\\
        \ifdefined\authormail
            \href{mailto:\authormail}{\authormail}\\
        \else \\ \fi
        \ifdefined\matrikelnr
            {Matrikelnummer: \matrikelnr\\}
        \else \\ \fi
        \ifdefined\authoraddress
            \authoraddress\\
        \else \\ \fi
        \end{flushleft}
        
        %\end{adjustwidth}
        \selectlanguage{\mainlanguage}
        \end{titlepage}
        
        \newpage
    }
\makeatother

%%%%%%%%%%%%%%%%%%Simple Title%%%%%%%%%%%%%%%%%%
\makeatletter         
    \renewcommand\maketitle{
        \selectlanguage{ngerman}
        \begin{flushleft}
            \@author\\
            \coursetype~\emph{\coursetitle}\\
            Kursnummer: \coursenumber\\
            \tutor\\
            \handindate
        \end{flushleft}
        
        \begin{center}
            \vspace{2em}
            \textsc{\huge{\@title}}\\[4ex]
        \end{center}
        \selectlanguage{\mainlanguage}
    } % Note the extra }
\makeatother

%%%%%%%%%%%%%%%Table of Contents%%%%%%%%%%%%%%%%%
\let\tableofcontentsORIG\tableofcontents
\renewcommand\tableofcontents{\thispagestyle{empty}\tableofcontentsORIG\clearpage}

%%%%%%%%%%%%%%%Works Cited%%%%%%%%%%%%%%%%%%%%%%%

    \newcommand{\workscited}{
        \newpage
        \phantomsection
        \addcontentsline{toc}{section}{Works Cited}
        \printbibliography
        }

%%%%%%%%%%%%%%%Plagiatserklärung%%%%%%%%%%%%%%%%%
\newcommand{\plagiat}{
    \newpage
    
    \selectlanguage{ngerman}
    \thispagestyle{empty}
    
    \section*{Plagiatserklärung}
    \pdfbookmark[0]{Plagiatserklärung}{Plagiatserklärung}
    
    \setlength{\parindent}{0em}
    
    \vspace*{20pt}
    Hiermit erkläre ich, dass ich die vorliegende Arbeit selbständig verfasst und keine anderen als die angegebenen Quellen und Hilfsmittel benutzt habe. 
    
    \vspace*{65pt}
    
    \ifdefined\city
        {\city, den} 
    \fi
    \ifdefined\handindate
        \handindate 
    \else \today 
    \fi \hfill \author
    }